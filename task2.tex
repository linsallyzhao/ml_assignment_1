\documentclass{article}
\usepackage[left=2cm,right=2cm,top=1cm,bottom=1cm,a3paper]{geometry}
\usepackage{pgfplots}
\usepackage{parskip}
\usepackage{caption}
\usepackage{multicol}
\usepackage{wrapfig}
\usepackage[]{mathtools}
\usepackage{float}
\restylefloat{table}

\setlength{\parindent}{0mm}

\pgfplotsset{compat=1.15}

\newcommand{\cia}[1]{\resizebox{\textwidth}{!}{\input{#1}}}
\newcommand{\ciapdf}[1]{\vspace*{-\parskip}\begin{center}\resizebox{0.75\textwidth}{!}{\includegraphics[width=0.75\textwidth]{#1}}\end{center}}
\newcommand{\halfhalf}[2]{\begin{multicols}{2}\includegraphics[width=0.5\textwidth]{#1} \includegraphics[width=0.5\textwidth]{#2}\end{multicols}}

\begin{document}

\title{Supervised learning task 2}
\author{Lin Zhao (16010906)}
\date{\today}
\maketitle

\newpage

\subsection*{Introduction}

Markovitz pertfolio theory has always be a classic. However it is known that
the optimal portfolio is not stable bringing issue on frequently rebalancing
for investors. Following the lead of Fastrich, Paterlini, and Winker (Constructing
optimal sparse portfolios using regularization methods, 2014), this study
try to find global minimum variance portfolio under regularizations thus stablize
the portfolio choice and reducing the rebalancing problem.

The regularizations used in this study are Lasso and Ridge. Resulting global
minimum variance portfolios has lower variance than Markovitz portfolios and
therefore confirmed the results from paper montioned above.

\subsection*{Data description and cleaning}

The data used is French 48 industry portfolios daily data. After comparing variance
of each industry and of the whole dataset, I choose to use equal weighted data
since it has higher variance and hopefully the different among regularizations
will be more obvious. After deleting any row that contains missing value, I have
about 44 years data left which is a good amount.

In figure 1, I plotted standard diviation of all 48 industies and we can see that
except for a few outliars, most industries standard diviation of returns lies
in a interval between 0.75 to 1.5.

\ciapdf{Figure_1T2.pdf}

\begin{quote}
Figure 1: std of 48 industires show that except for a few, all industries has
std with a steady interval
\end{quote}

Next I plottd mean returns for each of 48 industies together with minimum and maxmum
returns of each industry. The maxmum returns seem to has more fluctuation than
the minimum.

\ciapdf{Figure_2T2.pdf}

\begin{quote}
Figure 2: mean, minimum and maxmum returns of each industry
\end{quote}

The industry that has highest variance is the coal and utililies has the lowest
variance. Here is a plot to demonstrate that.

\ciapdf{Figure_3T2.pdf}

\begin{quote}
Figure 3: Variances of coal industry and utility industry.
\end{quote}

\subsection*{Global minimus variance portfolio with different regulariztions}

\subsubsection*{Three methods to select portfolios}

In this study, I used three methods for selecting minimum variance portfolios.
The frist one is Markovitz portfolio selection. This is a problem minimize variance
of the portfolio under the constraint that sum of all the weights equals to 1.
I also performed selection for long-only portfolios to use together with long-short
portfolio as benchmarks. To find the long-only portfolios, the minimize problem
need a extra constraint that all the weights greater or equal to 0. The second
method I used is a Markovitz under Lasso regularization. This is a minimizing
problem that minimizing sum of variance and a 1-norm penalty:

\begin{equation*}
\mathbf{w^T\sum w} + \lambda \sum_{n= 1}^{K}\left | w_n \right |,
\end{equation*}

where \textbf{w} is weight vector, $\mathbf{\Sigma}$ is the covariance matrix
and K is the number of stocks in your investing universe. Parameter $\lambda$
controls how much penalty to add. To find a portfolio, one need to
solve this minimizing problem under the same constrain that sum of weights equals
to one. The other method in this report is a Markovitz under Ridge regularization.
In this minimizing problem, the target function is a sum of variance and 2-norm
penalty:

\begin{equation*}
\mathbf{w^T\sum w} + \lambda \sum_{n= 1}^{K}\left | w_n \right |^2,
\end{equation*}

and solving this under same budget constraint give a minimum variance portfolio
under Ridge regularization.

\subsubsection*{Selection $\lambda$ with cross validation}

If $\lambda$ equals to 0, both regularized method collaps to classic Markovitz.
On the other hand, if $\lambda$ is too large, which means the penalty term in
the target function will dominate the variance term and thus impose to much bias.
To select optimal $\lambda$, I performed a cross validation as did in the Fastrich
paper. This cross validation is only meaningful under stationary assumption.
First, I shuffle the whole training set and divide it into 10 equal folds.
In each round, I take one of these 10 folds as test set and the rest 9 folds as
training set. Then, in each round, for each value in a series of $\lambda$
that I want to test,
solve the minimizing problem under relative constraint and thus find a portfolio.
Hold this portfolio in the test set and recieve a return series. Calculate the
standard diviation of this return series since our aim is to find minimum
variance portfolio. After 10 round, one will have 10 standard diviation for each
$\lambda$, and taking mean of these gives the score of that $\lambda$. The
optimal $\lambda$ is the one with the lowest score. And a good searching
interval should give a smile shape curve when plot all the score against $\lambda$.
If the optimal $\lambda$ is on the edge of searching interval, a change of interval
is required to find a true minimum score. In the next to figure, I plotted
scores against $\lambda$ for a Lasso penalty and a Ridge penalty.



\ciapdf{Figure_4T2.pdf}

\begin{quote}
Figure 4: portfolio std with Lasso regularizaton. Training period 1 year. Searching
interval $10^{-2}$ to $10^{-1.3}$ and searching steps 100.
\end{quote}


\ciapdf{Figure_5T2.pdf}

\begin{quote}
Figure 5: portfolio std with Ridge regularizaton. Training period 1 year. Searching
interval $10^{-2}$ to $10^{-1.3}$ and searching steps 100.
\end{quote}

Since there is randomness involved in the shuffling step, to find the reasonable
searching interval, I run cross validaton for both regularizatons multiple times
to make sure all the optimal values are with in the interval. When using on
the whold data set, if the optimal solution is on the edge, the program will
through a warning, shift both boundaries of interval to the coresponding
direction, and perform another search.

\subsection*{Out-of-sample test}

Out-of-sample tests are performed on the data with vary training windows and
holding periods. In the plot below is return series trained on a one year
window and hold for 3 months.


\ciapdf{Figure_6T2.pdf}

\begin{quote}
Figure 6: return series trained on 250 days and hold for 63 days. Cross validation
interval, Lasso: $10^{-2.1}$ to $10^{-1.2}$, Ridge: $10^{-2.3}$ to $10^{-1.3}$.
Searching steps: 100 for both.
\end{quote}

The variance of portfolios are 4.5392, 3.9004, 3.6182 and 3.3498 in percentage
respectly
for Markovitz long-short, Markovitz long-only, Lasso, and Ridge. As expected,
portfolio found under regularizations have lower variance during the holding
period. This is due to the barrier set by penalty term blocks the noise in the
data, thus in the variance covariance matrix, to enter the results freely. A
small surprise here is that with more constraint, the long-only portfoio
has lower variance than the long-short portfolio. This is not consistant across
all the back tests I performed, but it indicates the instability of Markovitz
solution.

Next, I performed a series of out-of-sampel tests with moving windows. I tested
different training window to see whether having longer training window affects
any methods. I also test different holding periods to see whether some methods
perform better when the selected portfolios were rebalanced more often. Since
Fastrich paper only used 5 to 6 years of data, and considering runing time, I
dicided to run these tests on the last 5 years from the whole data set.


\subsubsection*{\centering{}Out-of-sample test result}

\begin{table}[H]
    \caption{Out-of-sample test with moving windows. Data use are the last 1250
    observations of the data set}
    \begin{center}\begin{tabular}{|l|l|l|l|l|l|l|}
    \hline
    Training window (days) & Holding window (days) & window number & Marlovitz long-short & Markovitz long-only & Lasso & Ridge \\ \hline
    500 & 63 & 11  & 3.5831 & 4.3738 & 3.4346 & 3.4377 \\ \hline
    250 & 63 & 15  & 3.4866 & 4.3458 & 3.2998 & 3.2627 \\ \hline
    125 & 63 & 17  & 4.1919 & 4.3288 & 3.4325 & 3.4712 \\ \hline
    250 & 125 & 8  & 3.9520 & 4.5060 & 3.6145 & 3.7762 \\ \hline
    250 & 63 & 15  & 3.4866 & 4.3459 & 3.2840 & 3.2682 \\ \hline
    250 & 21 & 47  & 3.3769 & 4.1968 & 3.1406 & 3.1526 \\ \hline
    \end{tabular}\end{center}
\end{table}

From the whole table, we can see that the result is robust to the change of
training table and holding period. In all the tests cross various training period
and holding length, the moving window out-of-sample tests denmonstrate that
regularized portfolio constently out perform the Markovitz portfolio. Among
two bench marks, long-only protfolio has higher variance because it cannot
take adventage of shorting thus has less ability to diverge the risk. Between
Lasso and Ridge, there is no clear result about which one provide portfolios
with lower variance. The first three row of the table seems to indicate that
shorter training period make the regularized portfolio out perform even more.
This is not enough to evidence to conclude, but one potential reason could be
that in a short training period, the true information in the covariance matrix
is more mingled with noise and thus the Markovitz method has more difficulty
to select real optimal portfolio. And because of the penalty term, regularized
methods has a more restrict standard for only the significant information to
pass through, thus allow them to out perform the Markovitz even more. For the
changing of holding period, there is no clear result about whether a certain
length of holding benifit a particular method.

At last, I performed a out-of-sample test on the whole 44 years data. Resulting
variance are 2.1290, 3.5056, 2.0711, and 2.1074 for Markovitz long-short,
Markovitz long-only, Lasso, and Ridge. As usual, portfolios under regulirizations
out performed both benchmarks. The order of variances for the four portfolios
and magnitude of differences
match the results of Fastrich, Paterlini and Winker paper.

Both Fastrich paper and Brodie paper mentioned sparseness as another feature as
portfolio chosen by regularized Markovitz methods, however, set threshold at 0,
neither set of portfolios selected under regularizations show this feature. I
did not expore more from here but it might related to the solver used. Also,
a position with weight less than $10^{-2}$ is practically non-active, so the
sparse feature can also rely on the choice of threshold.

\subsection*{Conclusion and potential further questions to answer}

Observed global minimum variance portfolios found under different regularizations,
I confirmed that regularized protfolios out perform both Markovitz benchmarks
in the out-of-sample tests thus found optimal portfolios has lower variance in various
the holdidng periods. Markovitz solutions are not stable and contains noice from the
data so regularization provides practical benefit to investors. Another benefit
of regularizations proposed by both Fastrich paper and Brodie paper is it gives
sparse portfolios thus reduce the transition cost. This is not observed here
and can be a interesting further topic.

\begin{thebibliography}{9}
\bibitem{booms} 
B.Fastrich, S.Paterlini \& P.Winker (2014)
\textit{Constructing optimal sparse portfolios using regularization methods}.

\bibitem{fcp}
J.Brodie et al. (2009)
\textit{Sparse and stable Markovitz portfolios}.
\end{thebibliography}


\end{document}
